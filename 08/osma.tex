\documentclass{article}

\usepackage[utf8]{inputenc}
\usepackage{amsmath}
\usepackage{mathtools}
\usepackage{graphicx}
\usepackage{subcaption}
\usepackage{caption}
\usepackage{float}
\usepackage[slovene]{babel}
\usepackage{geometry}
\usepackage{physics}
\usepackage{mathrsfs}
\geometry{margin=1in}

\title{Metoda končnih elementov: lastne rešitve}
\author{Andrej Kolar-Požun, 28172042}



\errorcontextlines 10000
\begin{document}
\pagenumbering{gobble}
\maketitle
\pagenumbering{arabic}
\section{Polkrožna cev}

Z metodo končnih elementov bomo tokrat reševali Helmholtzovo enačbo $\nabla^2 u +k^2 u = 0$. V šibki formulaciji se ta glasi
\begin{equation*}
S(u) = \frac{1}{2} \langle \nabla u, \nabla u \rangle -  \frac{1}{2}\lambda\langle u,u \rangle
\end{equation*}
Kjer skalarni produkt definiramo na sledeč način
\begin{equation*}
\langle g,h \rangle = \int_\Omega dx g(x) h(x) 
\end{equation*}

Po metodi Galerkina rešitev naše enačbe zapišemo kot linearno kombinacijo trial funkcij
\begin{equation*}
u = \sum_{i=1}^N a_i w_i
\end{equation*}
Za funkcije $w_i$ bomo spet vzeli funkcijo, ki je v $x_i$ enaka ena, na sosedih enaka nič, vmes pa linearno narašča.

Če zgornji funkcional variramo dobimo:
\begin{align*}
&\sum_{j=1}^N \left(A_{ij} - \lambda B_{ij} \right) a_j = 0 \\
&A_{ij} = \langle \nabla w_i , \nabla w_j \rangle \\
&B_{ij} = \langle w_i , w_j \rangle
\end{align*}

Spet bomo domeno trinagulirali. Matrika A je enaka kot prej, novi so le izvendiagonalni elementi matrike B, ki se glasijo 
\begin{equation*}
B_{ij} = \sum S_r / 12
\end{equation*}
Kjer gre vsota po vseh trikotnikih na zveznici i-j.



Najprej bom narisal nekaj slik, pri katerih sem povsod risal z 200 točkami, od tega jih je četrtina bila na robu. Triangulacijo sem delal z scipy-jevo metodo Delaunay, matrični sistem pa na koncu rešil z linalg.solve
Poiseuillov koeficient sem računal tako, da sem povprečil vrednost pretoka po vseh 3 ogliščih trikotnika, pomnožil s ploščino trikotnika in to seštel po vseh trikotnikih.

Delal bom z točkami porazdeljenimi enakomerno po polkrogu:

\begin{figure}[H]
\centering
\begin{subfigure}{.49\textwidth}
\includegraphics[width=\linewidth]{prva/1.png}
\end{subfigure}
\begin{subfigure}{.49\textwidth}
\includegraphics[width=\linewidth]{prva/2.png}
\end{subfigure}
\caption*{}
\end{figure}


\begin{figure}[H]
\centering
\begin{subfigure}{.49\textwidth}
\includegraphics[width=\linewidth]{prva/3ups.png}
\end{subfigure}
\begin{subfigure}{.49\textwidth}
\includegraphics[width=\linewidth]{prva/3.png}
\end{subfigure}
\caption*{}
\end{figure}

\begin{figure}[H]
\centering
\begin{subfigure}{.49\textwidth}
\includegraphics[width=\linewidth]{prva/4.png}
\end{subfigure}
\begin{subfigure}{.49\textwidth}
\includegraphics[width=\linewidth]{prva/5.png}
\end{subfigure}
\caption*{}
\end{figure}

\begin{figure}[H]
\centering
\begin{subfigure}{.49\textwidth}
\includegraphics[width=\linewidth]{prva/6.png}
\end{subfigure}
\begin{subfigure}{.49\textwidth}
\includegraphics[width=\linewidth]{prva/7.png}
\end{subfigure}
\caption*{}
\end{figure}

\begin{figure}[H]
\centering
\begin{subfigure}{.49\textwidth}
\includegraphics[width=\linewidth]{prva/primerjava.pdf}
\end{subfigure}
\begin{subfigure}{.49\textwidth}
\includegraphics[width=\linewidth]{prva/primerjava2.pdf}
\end{subfigure}
\caption*{Primerjava izračunanih frekvenc z ničlami Besselovih funkcij(analitična rešitev). Na desni vidimo, da relativna napaka počasi raste, najbrž bi jo lahko zmanjšal z izbiro večje matrike.}
\end{figure}
\section{Metoda Galerkina}

Isto nalogo bomo zdaj rešili še samo z metodo Galerkina.
Naše poskusne funkcije so:
\begin{equation*}
w_{m,k} (r, \varphi) = r^{m+k}(1-r)sin(m \varphi)
\end{equation*}
Kjer sta m in k naravni števili(k je lahko tudi nič). Takšne funkcije vse zadoščajo našim robnim pogojem.

Rešujemo torej spet $\nabla^2 u - k^2 u = 0$
Delali bomo kar na kvadratni mreži, kjer so integrali preprostejši.
Analitično lahko dobimo:
\begin{align*}
&A_{(m,k),(n,l)} = \int r dr d\varphi \  \nabla w_{mk} \nabla w_{nl} \\
&A_{(m,k),(n,l)} = \delta_{mn} \frac{\pi}{2} \frac{(1+2m)(l+2m) + k(1+2l+2m)}{(k+l+2m)(1+k+l+2m)(2+k+l+2m)}\\
&B_{(m,k),(n,l)} = \int r dr d\varphi  \ w_{mk} w_{nl} \\
&B_{(m,k),(n,l)} = \delta_{mn}\frac{\pi}{2} \left(\frac{1}{2m+k+l+2} - \frac{2}{2m+k+l+3} + \frac{1}{2m+k+l+4} \right)
\end{align*}

Sedaj pa še nekaj slik, kjer sem narisal 10000 točk. Številke m in k v naslovih povejo do kje sem vzel bazne funkcije.

\begin{figure}[H]
\centering
\begin{subfigure}{.49\textwidth}
\includegraphics[width=\linewidth]{druga/1.png}
\end{subfigure}
\begin{subfigure}{.49\textwidth}
\includegraphics[width=\linewidth]{druga/2.png}
\end{subfigure}
\caption*{}
\end{figure}

\begin{figure}[H]
\centering
\begin{subfigure}{.49\textwidth}
\includegraphics[width=\linewidth]{druga/3.png}
\end{subfigure}
\begin{subfigure}{.49\textwidth}
\includegraphics[width=\linewidth]{druga/4.png}
\end{subfigure}
\caption*{}
\end{figure}

\begin{figure}[H]
\centering
\begin{subfigure}{.49\textwidth}
\includegraphics[width=\linewidth]{druga/5.png}
\end{subfigure}
\begin{subfigure}{.49\textwidth}
\includegraphics[width=\linewidth]{druga/6.png}
\end{subfigure}
\caption*{}
\end{figure}


Izgleda, da metoda deluje dobro, tudi pri nižjih m in k. Poskusil bom to preveriti še malo bolj sistematično:

\begin{figure}[H]
\centering
\begin{subfigure}{.49\textwidth}
\includegraphics[width=\linewidth]{druga/primerjava.pdf}
\end{subfigure}
\begin{subfigure}{.49\textwidth}
\includegraphics[width=\linewidth]{druga/primerjava2.pdf}
\end{subfigure}
\caption*{Na levi vidimo, da lastna vrednost sploh ni odvisna od števila m, če je to le več kot 0. Na desni pa vidimo, da so relativne napake spet zelo majhne a ne naraščajo kot prej, ampak imajo bolj nepravilno odvisnost.}
\end{figure}

Poglejmo kako se obnese slabša izbira trial funkcij:
\begin{align*}
&w_{m,k}(r,\varphi) = sin(k\pi r/R) sin(m \varphi) \\ 
& \nabla w_{m,k} = \frac{\pi k}{R} cos(k \pi r/R) sin(m \varphi) + \frac{sin(k \pi r/R)}{r} m cos(m \varphi)
\end{align*}

Skalarne produkte sem pri teh funkcijah računal kar numerično z Gaussovo kvadraturo(Čeprav verjetno obstaja tudi analitična formula).

\begin{figure}[H]
\centering
\begin{subfigure}{.49\textwidth}
\includegraphics[width=\linewidth]{druga/sin.png}
\end{subfigure}
\begin{subfigure}{.49\textwidth}
\includegraphics[width=\linewidth]{druga/sin2.png}
\end{subfigure}
\caption*{Tudi taka izbira baznih funkcij kar v redu deluje!}
\end{figure}

\begin{figure}[H]
\centering
\begin{subfigure}{\textwidth}
\includegraphics[width=\linewidth]{druga/sinprimerjava.pdf}
\end{subfigure}
\caption*{Slika je malce drugačna kot prej. Napak se ustali šele pri m=2, z k-jem pa se seveda manjša, vendar ostane večja kot pri prejšnjih funkcijah. }
\end{figure}

\begin{figure}[H]
\centering
\begin{subfigure}{.49\textwidth}
\includegraphics[width=\linewidth]{druga/sinprimerjava2.pdf}
\end{subfigure}
\begin{subfigure}{.49\textwidth}
\includegraphics[width=\linewidth]{druga/sinprimerjava3.pdf}
\end{subfigure}
\caption*{Pri m=k=5 je napaka kar velika in se veča. Pri m=k=10 pa zgleda stvar bolj v redu, vidimo tudi, da imajo outlierji manjšo napako kot prej.}
\end{figure}

Za na konec sem preizkusil še trial funkcije v radialni smeri izbrati kot piramidalne funkcije, ki so 0 na robu, 1 na k/10 ter linearne vmes. Integrale sem spet izračunal numerično v kotni odvisnosti pa sem obdržal sinus. 

\begin{figure}[H]
\centering
\begin{subfigure}{\textwidth}
\includegraphics[width=\linewidth]{druga/idk.png}
\end{subfigure}
\caption*{}
\end{figure}

\begin{figure}[H]
\centering
\begin{subfigure}{\textwidth}
\includegraphics[width=\linewidth]{druga/idk2.pdf}
\end{subfigure}
\caption*{Izgleda, da se le za prvo lastno vrednost dobro izide.}
\end{figure}
\end{document}


\begin{figure}[H]
\centering
\begin{subfigure}{\textwidth}
\includegraphics[width=\linewidth]{prva/random.pdf}
\end{subfigure}
\caption*{Slika je malce drugačna kot prej. Napak se ustali šele pri m=2, z k-jem pa se seveda manjša, vendar ostane večja kot pri prejšnjih funkcijah. }
\end{figure}