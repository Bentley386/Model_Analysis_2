\documentclass{article}

\usepackage[utf8]{inputenc}
\usepackage{amsmath}
\usepackage{mathtools}
\usepackage{graphicx}
\usepackage{subcaption}
\usepackage{caption}
\usepackage{float}
\usepackage[slovene]{babel}
\usepackage{geometry}
\usepackage{physics}
\usepackage{mathrsfs}
\geometry{margin=1in}

\title{Navadne diferencialne enačbe: robni problemi}
\author{Andrej Kolar-Požun, 28172042}



\errorcontextlines 10000
\begin{document}
\pagenumbering{gobble}
\maketitle
\newpage
\pagenumbering{arabic}
\section{Oblika vrvi}

Zanima nas oblika vrvi, ki je obešena na dveh točkah na vrteči se navpični osi.
Obliko opisuje stacionarna valovna enačba
\begin{align*}
\frac{d}{ds}\left( F \frac{dx}{ds}\right) + \rho \omega^2 x &= 0 \\
\frac{d}{ds}\left( F \frac{dy}{ds}\right) - \rho g &= 0 \\
\left( \frac{dx}{ds}\right) ^2 + \left( \frac{dy}{ds}\right) ^2 &= 1
\end{align*}

Zmanjšajmo število prostih parametrov s substitucijami $s \rightarrow s/l$, $x \rightarrow x/l$, $F \rightarrow F/\rho g l$, $dx/ds = cos \alpha$, $dy/ds = sin \alpha$, $\beta = \omega^2 l /g$. Dobimo:
\begin{align*}
\frac{dF}{ds} &= - \beta x cos \alpha +  sin \alpha \\
F \frac{d \alpha}{ds} &=  \beta x sin \alpha + cos \alpha \\
\frac{dx}{ds} & = cos \alpha \\
\frac{dy}{ds} &= sin \alpha
\end{align*}

Z robnimi pogoji $x(s=0) = y(s=0) = 0$ in $x(s=1) = 0, y(s=1) = -1$. Naloge se bom lotil z metodo Runge Kutta z adaptivnim korakom, ki jo bom pognal pri nekih začetnih pogojih $\alpha(s=0)$ in $F(s=0)$ ter z funkcijo fsolve poiskal pravilna začetna pogoja, da bo robnim pogojem zadoščeno. 

\begin{figure}[H]
\centering
\begin{subfigure}{.49\textwidth}
\includegraphics[width=\linewidth]{prva/vecbet.pdf}
\end{subfigure}
\begin{subfigure}{.49\textwidth}
\includegraphics[width=\linewidth]{prva/vecbet2.pdf}
\end{subfigure}
\caption*{Prvi rezultati za predstavi, risal sem samo za večje koeficiente beta, ker je tam bolj zanimivo. $x_0^0$ pomeni začetni približek(v fsolve funkciji) za začetno vrednost spremenljivke x. Dejanska vrednost teh parametrov je za vsako rešitev drugačna in jo je našla metoda.}
\end{figure}

\begin{figure}[H]
\centering
\begin{subfigure}{.49\textwidth}
\includegraphics[width=\linewidth]{prva/vecbet3.pdf}
\end{subfigure}
\begin{subfigure}{.49\textwidth}
\includegraphics[width=\linewidth]{prva/vecbet4.pdf}
\end{subfigure}
\caption*{Še nekaj primerov. Vidimo, da je možnih rešitev zelo veliko, zato jih bom raje predstavil na prikladnejši način z animacijo. Mimogrede ne pozabimo, da so rešitve prezrcaljene čez y os ekvivalentne.}
\end{figure}


Za več primerov oblik glej animacijo prva.mp4, na kateri je za več različnih parametrov prikazana oblika vrvi in odvisnost sile F(s). Zaradi števila različnih prikazanih kombinacij parametrov je priporočljivo, da se animacijo po potrebi upočasni ali pavzira.
Na animaciji sem skušal spremljati tudi odvisnost maksimalnih odmikov vrvi od frekvence vrtenja, kar se ni najbolj idealno izšlo, saj metoda pod različno frekvenco morda konvergira k drugi rešitvi in tako ne morem narisati le odvisnosti od frekvence ampak je zmeraj nekje od zadaj še izbira začetne sile in kota.
Legenda za animacijo:
\begin{description}
  \item[$\bullet$ $\beta$] Parameter problema, sorazmeren z $\omega^2$
  \item[$\bullet$ $F_0^0$] Začetni približek za silo v točki s=0
  \item[$\bullet$ $\alpha_0^0$] Začetni približek za kot v točki s=0
  \item[$\bullet$ $F_0$] Dejanska sila v točki s=0
  \item[$\bullet$ $\alpha_0$] Dejanski kot v točki s=0
  \item[$\bullet$ $y_1$] y koordinata spodnjega prijemališča
\end{description}
\newpage
\section{Gibanje zvezde skozi galaksijo}

Gibanje zvezd skozi galaksijo opiše naslednji potencial(Henon-Heiles)
\begin{equation*}
U(x,y) = \frac{1}{2}(x^2+y^2)+x^2 y - \frac{1}{3}y^3
\end{equation*}
Iz česar sledijo enačbe gibanja(maso postavimo na ena):
\begin{align*}
\ddot{x} &= -x - 2xy\\
\ddot{y} &= -y - x^2 +y^2
\end{align*}

Sistem pretvorimo v sistem DE prvega reda:
\begin{align*}
\dot{v} &= -x - 2xy\\
\dot{u} &= -y -x^2 + y^2\\
\dot{x} &= v \\
\dot{y} &= u
\end{align*}

Integrator bom pognal z začetnim pogojem $x(0)=0, y(0) \in [0,1]$, Začetne hitrosti bom streljal, s tem, da bo veljalo $E<\frac{1}{6}$

\begin{figure}[H]
\centering
\begin{subfigure}{\textwidth}
\includegraphics[width=\linewidth]{druga/test1.pdf}
\end{subfigure}
\caption*{Trajektorije pri treh časih za zgornje parametre. $v_0$ določim iz energije. Orbita na sliki ni periodična.}
\end{figure}

\begin{figure}[H]
\centering
\begin{subfigure}{\textwidth}
\includegraphics[width=\linewidth]{druga/test2.pdf}
\end{subfigure}
\caption*{Malo drugačna(spet neperiodična) slika. Ker sem tako razliko dobil le s spremembo enega izmed treh parametrov, bom spet orbite raje prikazal z prikladno animacijo.}
\end{figure}

V animaciji y0.mp4 sem pri fiksinih parametrih spreminjal začetno lego y. Dobimo lepe vzorce, podobne zgornjim


V animaciji u0.mp4 sem spreminjal zacetno vrednost hitrosti v y smeri. Na približno $\delta u_0 = 0.15$ širokem območju dobimo novo obliko orbite, ki še vedno izgleda kar pravilna.

V animaciji E0.mp4 pa sem spreminjal energijo in vidimo, da dobimo, ko se približujemo mejni vrednosti 1/6 kaotično zgledajoče trajektorije.

Poglejmo si pobliže na primer ta primer:
\begin{figure}[H]
\centering
\begin{subfigure}{\textwidth}
\includegraphics[width=\linewidth]{druga/kaos.pdf}
\end{subfigure}
\end{figure}
Očitno se tu dogaja nekaj neintuitivnega. Kot je pogosto v navadi v teoriji dinamičnih sistemov si bomo izbrali sečno ploskev in na njej pogledali fazni portret. Ta sečna ploskev bo $x=0$. Z izbiro sečne ploskvije nekako pretvorimo problem iz 3 dimenzionalnega v le 2 dimenziji(originalno je sicer 4 dimenzionalen, a nam ohranitev energije lahko zmanjša število parametrov).

\begin{figure}[H]
\centering
\begin{subfigure}{\textwidth}
\includegraphics[width=\linewidth]{druga/poincare.pdf}
\end{subfigure}
\caption*{Fazni portret na sečni ploskvi. Možna težava se pojavi, ker v numeriki gledamo mi le diskretne x in tako ne moremo slediti točno, kdaj pridemo na x=0 in se moramo zadovoljiti z določenim maksimalnim odstopanjem od x=0. V kodi sem tukaj na roke zahteval, da morajo biti 2 sledeči si x=0 točki za vsaj t=10 narazen(da ne bi večkrat štel iste točke, ki se počasi premika), s čemer sem lahko zgrešil kako izmed njih.}
\end{figure}

\begin{figure}[H]
\centering
\begin{subfigure}{\textwidth}
\includegraphics[width=\linewidth]{druga/poincare2.pdf}
\end{subfigure}
\caption*{Tukaj sem izbral več točk, ki so se po začetni legi razlikovale za tisočinko. Upal sem, da bo po dolgem času razlika zelo opazna, kar je značilnost kaosa. Nisem prepričan, če je prikazano dovolj velika razlika, da bi lahko zaključili, da gre res za kaos.}
\end{figure}

Bolj sistematičen pristop k temu, da povemo ob katerih pogojih pride do kaotičnega obnašanja in koliko kaotično je to obnašanje bi bil, da bi za sistem numerično izračunali največji Lyapunov eksponent, katerega predznak bi nam povedal ali smo v kaotičnem režimu ali ne.


\end{document}