\documentclass{article}

\usepackage[utf8]{inputenc}
\usepackage{amsmath}
\usepackage{mathtools}
\usepackage{graphicx}
\usepackage{subcaption}
\usepackage{caption}
\usepackage{float}
\usepackage[slovene]{babel}
\usepackage{geometry}
\usepackage{physics}
\usepackage{mathrsfs}
\geometry{margin=1in}

\title{Direktno reševanje Poissonove enačbe}
\author{Andrej Kolar-Požun, 28172042}



\errorcontextlines 10000
\begin{document}
\pagenumbering{gobble}
\maketitle
\pagenumbering{arabic}
\section{Poves opne}

V tej nalogi bomo izkoristali dejstvo, da znamo nekatere enačbe v določenih preprostih geometrijah in z preprostimi robnimi pogoji delno ali v celoti analitično rešiti.
Začnimo z neobteženo kvadratno opno, rešujemo enačbo(v ustreznih enotah)
\begin{equation*}
\nabla^2 u(x,y) = -p(x,y)
\end{equation*}
Z Dirichletovimi robnimi pogoji $u(x,y)=0$ na robu.

Z izbiro sinusne Fouriere transformacije bomo robne pogoje že takoj upoštevali:
\begin{align*}
&G^{mn} = \sum_{j=1}^{J-1}\sum_{l=1}^{L-1} g_{jl} \sin \frac{\pi j m}{J} \sin \frac{\pi l n}{L} \\
&U^{mn} = \sum_{j=1}^{J-1}\sum_{l=1}^{L-1} u_{jl} \sin \frac{\pi j m}{J} \sin \frac{\pi l n}{L} \\
\end{align*}

Vstavimo v enačbo(Laplacian zapišemo v diskretni obliki) in dobimo:
\begin{equation*}
U^{mn} = h^2/2 \frac{G^{mn}}{\cos(\pi m/J) + \cos(\pi n / L) - 2}
\end{equation*}

Rešitev dobimo z inverzno sinusno transformacijo:
\begin{equation*}
u_{jl} = \frac{2}{J}\frac{2}{L} \sum_{m=1}^{J-1} \sum_{n=1}^{L-1} U^{mn} \sin \frac{\pi j m}{J} \sin \frac{ \pi l n }{L}
\end{equation*}

Še ena možnost je, da Fourierovo transformacijo napravimo le v eni izmed dimenzij(izbral si bom x), v drugi nam potem ostane preprosta diferenčna shema
\begin{align*}
&U^m_l = \sum_n u_{ln} \sin \frac{\pi n m}{N} \\
&G^m_l = \sum_n g_{ln} \sin \frac{\pi n m}{N} \\
&U_{l-1}^m - U_l^m (4- 2 \cos(m \pi/J)) + U_{l+1}^m = h^2 G_l^m 
\end{align*}

\begin{figure}[H]
\centering
\begin{subfigure}{.5\textwidth}
\includegraphics[width=\linewidth]{kalibracija.pdf}
\end{subfigure}
\caption*{Najprej na hitro preverimo kakšen je optimalen omega za SOR v tem primeru. Rezultat po 50 iteracijah primerjamo s tistim, ki smo ga dobili tako, da smo iterirali dokler ni bil rezidum 0.0001(pribl 500 iteracij)}
\end{figure}

\begin{figure}[H]
\centering
\begin{subfigure}{.49\textwidth}
\includegraphics[width=\linewidth]{1dfour1000.png}
\end{subfigure}
\begin{subfigure}{.49\textwidth}
\includegraphics[width=\linewidth]{2dfour1000.png}
\end{subfigure}
\caption*{}
\end{figure}

\begin{figure}[H]
\centering
\begin{subfigure}{.5\textwidth}
\includegraphics[width=\linewidth]{SOR100.png}
\end{subfigure}
\caption*{SOR je bil tako počasen, da sem ga raje izvedel na manjši mreži. Vidimo, da naše 3 metode dajo smiselen rezultat.}
\end{figure}

\begin{figure}[H]
\centering
\begin{subfigure}{.5\textwidth}
\includegraphics[width=\linewidth]{hitrosti.pdf}
\end{subfigure}
\caption*{Tukaj vidimo, zakaj se splača uporabiti Fourierovo metodo, kljub temu, da nam SOR da prave rezultate. Veliko hitrejša je, pa tukaj je bil zaustavitveni pogoj za iteracijo še majhen - norma 0.1 vektorja reziduala.}
\end{figure}

Poglejmo si še obteženo opno, kjer je ta obtežena po porazdelitvi mase kot v 6. nalogi(glej 6. nalogo za sliko)

\begin{figure}[H]
\centering
\begin{subfigure}{.49\textwidth}
\includegraphics[width=\linewidth]{obtezena/1dfour1000.png}
\end{subfigure}
\begin{subfigure}{.49\textwidth}
\includegraphics[width=\linewidth]{obtezena/2dfour1000.png}
\end{subfigure}
\caption*{}
\end{figure}

\begin{figure}[H]
\centering
\begin{subfigure}{.5\textwidth}
\includegraphics[width=\linewidth]{obtezena/sor100.png}
\end{subfigure}
\caption*{}
\end{figure}

\section{Temperaturni profil}

Spet rešujemo temperaturni profil v valju.
Če zapišemo Laplacian v cilindričnih koordinatah dobimo:
\begin{equation*}
\frac{\partial^2 u}{\partial r^2} + \frac{1}{r} \frac{\partial u}{\partial r} + \frac{\partial^2 u}{\partial z^2} = 0
\end{equation*}

Tisto deljenje z radijem je grdo, če bi se zadeve lotili z Fourierovo transformacijo v r smeri. Zato jo bomo delali le v z smeri, v r smeri pa obdržali diferenčno shemo. 

Naši robni pogoji so, da imamo določeno temperaturo $T_1$ na osnovnih ploskvah valja, ter temperaturo $T_2$ na plašču.
Za toplotni profil rešujemo Laplacevo enačbo $\nabla^2 T = 0$. Lahko definiramo novo spremenljivko $u = T - T_1$, za katero se enačba spet glasi $\nabla^2 u = 0$, s tem da imam sedaj za u homogeni robni pogoj v z smeri in lahko spet uporabimo sinusno transformacijo. Na plašlu bomo potem zahtevali temperaturo $T_2-T_1$ na koncu pa spet prešli na spremenljivko T $T = u  + T_1$

Dobimo:
\begin{align*}
&\frac{u^j_{k+1} + u^j_{k-1} - 2u^j_k}{h^2} + \frac{1}{kh} \frac{u^j_{k+1} - u^j_{k-1}}{2h} + \frac{u^j_k}{h^2}\left(2 \cos(\pi*j/N)-2\right) = 0 \\
& \left(2 \cos(\pi j /N) - 4 \right) u^j_k + \left(1 - \frac{1}{2k} \right) u^j_{k-1} + \left(1 + \frac{1}{2k} \right) u^j_{k+1} = 0  \\
&\tilde{u}^j_k = \frac{1}{2 \cos(\pi j /N) - 4}\left(-\left(1 - \frac{1}{2k} \right) u^j_{k-1} - \left(1 + \frac{1}{2k} \right) u^j_{k+1} \right) \\
&\tilde{u}^j_k = \frac{1}{2 \cos(\pi j /N) - 4}\left(-\left(1 - \frac{1}{2k} \right) u^j_{k-1} - \left(1 + \frac{1}{2k} \right) u^j_{k+1} \right) \\
&u^{j(n+1)}_k = u^{j(n)}_k + \omega \left( \tilde{u}^{j(n)}_k - u^{j(n)}_k \right)
\end{align*}

\begin{figure}[H]
\centering
\begin{subfigure}{.49\textwidth}
\includegraphics[width=\linewidth]{toplotni1.png}
\end{subfigure}
\begin{subfigure}{.49\textwidth}
\includegraphics[width=\linewidth]{toplotni2.png}
\end{subfigure}
\caption*{Slike pridejo takšne kot jih pričakujemo}
\end{figure}


Na hitro preverimo še če bi šlo z Neumannovim robnim pogojem. V tem primeru bi fiksiral tok na ploksvi in uporabil kosinusno transformacijo.
Nova spremenljivka bi se glasila $T = u + z* j$, kjer ima u ničelni odvod na robu. 


\section{Sinusna transformacija}

Scipyjeva funkcija za večdimenzionalno sinusno transformacijo mi je nekako dajala napačen rezultat in se zdela nekako čudno definirana.

Zato sem jo raje sprogramiral sam z učinkovitim algoritmom iz knjige Numerical Recipes, ki ga bom na kratko napisal tu:

Iz naših podatkov $f_j$ želimo dobiti sinusno transformiranko $F_k$. Najprej tvorimo novo zaporedje:
\begin{align*}
&y_0 = 0 \\
&y_j = \sin(j*\pi/N) (f_j + f_{N-j}) + 0.5(f_j - f_{N-j}) 
\end{align*}

Na zaporedju y uporabimo algoritem za realen FFT. Izkaže se, da je realni del te transformiranke enak $F_{2k+1} - F_{2k-1}$, imaginarni pa $F_{2k}$.
Tako takoj dobimo sode člene naše sinusne transformacije, za lihe pa prvega dobimo z uporabo dejstva $F_1 = -F_{-1}$ ostale pa iz le-tega rekurzivno.



\end{document}

\begin{figure}[H]
\centering
\begin{subfigure}{\textwidth}
\includegraphics[width=\linewidth]{prva/random.pdf}
\end{subfigure}
\caption*{Slika je malce drugačna kot prej. Napak se ustali šele pri m=2, z k-jem pa se seveda manjša, vendar ostane večja kot pri prejšnjih funkcijah. }
\end{figure}


