\documentclass{article}

\usepackage[utf8]{inputenc}
\usepackage{amsmath}
\usepackage{mathtools}
\usepackage{graphicx}
\usepackage{subcaption}
\usepackage{caption}
\usepackage{float}
\usepackage[slovene]{babel}
\usepackage{geometry}
\usepackage{physics}
\usepackage{mathrsfs}
\geometry{margin=1in}

\title{Parcialne diferencialne enačbe: robni problemi in relaksacija}
\author{Andrej Kolar-Požun, 28172042}



\errorcontextlines 10000
\begin{document}
\pagenumbering{gobble}
\maketitle
\pagenumbering{arabic}
\section{Poiseuillov koeficient}

Za reševanje parcialnih diferencialnih enačb z robnimi pogoji bom uporabil naslednje metode, vse povezane z diskretizacijo odvodov.
Reševal bom enačbo $\nabla^2 u = - q$

Za reševanje diskretiziranega sistema bom uporabil Jacobijevo in Gauss-Seidelovo iteracijo, kateri se glasita:
\begin{align*}
&u^{n+1}_{jk} = 0.25 \left( \sum_{nn} u^n + h^2 q_{jk}\right) \\ 
&u_{jk}^{n+1} = \frac{h^2}{4} \left( q_{jk} + \frac{1}{h^2} (u^{n}_{j+1,k} + u^{n+1}_{j-1,k}) + \frac{1}{h^2} (u^{n}_{j,k+1} + u^{n+1}_{j,k-1})\right)
\end{align*}

Zgornji enačbi v taki obliki veljata, če mrežo diskretiziram v x in y smeri z enakim odmikom h. Superscript n predstavlja n-ti približek za rešitev, j in k pa sta koordinati v diskretizirani mreži. Vsota po nn pomeni vsota po najbližjih sosedih mesta j,k.

Zgornji metodi bom poskusil izboljšati z metodo SOR, ki se zapiše kot:
\begin{equation*}
u_i^{n+1} = u_i^n + \omega(\tilde{u}_i^n - u_i^n)
\end{equation*}
Kjer je $\tilde{u}$ rešitev pridobljena z Jacobijevo oziroma Gauss-Seidelovo metodo.

Najprej bom metodo testiral na podobnem primeru, katerega sem enkrat že obravnaval in imam torej pravilne rešitve za referenco.
Takrat sem izračunal Poiseuillov koeficient za pretok skozi kvadratno cev:

Rešujem torej enačbo(v primernih enotah) 
\begin{equation*}
\nabla^2 u = - 1
\end{equation*}
Zanimal nas bo Poiseuillov koeficient, ki ga bom dobil kot
\begin{equation*}
C = \frac{8\pi}{S^2} \int_0^1 \int_0^1 u(x,y) dx dy
\end{equation*}

Spomnimo se, da za pravokotno cev s stranicama a in b velja enačba

\begin{equation*}
C = 2\pi \frac{b}{a}\Big( \frac{1}{3} - \frac{b}{a} \frac{64}{\pi^5} \sum_{n=1}^{\infty} \frac{tanh(\frac{a}{b} \frac{(2n-1)\pi}{2})}{(2n-1)^2}\Big)
\end{equation*}
Za kvadratno cev (a=b=1) nam to da 0.8832714348933981
Sedaj ko vemo pravo rešitev, primerjajmo hitrost različnih metod:

\begin{figure}[H]
\centering
\begin{subfigure}{.49\textwidth}
\includegraphics[width=\linewidth]{prva/hitrosti20.pdf}
\end{subfigure}
\begin{subfigure}{.49\textwidth}
\includegraphics[width=\linewidth]{prva/hitrosti40.pdf}
\end{subfigure}
\caption*{Chebyshev in Gaussov SOR sta zelo blizu, obe pa približno za red velikosti hitrejši od navadnega Gaussa in Jacobija. Za SOR je bil uporaljen domnevno optimalen $\omega$ iz navodil. S tako diskretizacijo mreže sem dobil $C \approx 0.83$}
\end{figure}

\begin{figure}[H]
\centering
\begin{subfigure}{.49\textwidth}
\includegraphics[width=\linewidth]{prva/hitrosti100.pdf}
\end{subfigure}
\caption*{Tukaj se lepo vidi prednost boljših algoritmov. Navaden Jacobi in Gauss v tem času(nekaj minut) sploh nista še prišla do rešitve, medtem ko sta ostali dve metodi že natančni. Tukaj sem dobil $C \approx 0.865$ kar je že bližje pravi vrednosti od redkejše mreže.}
\end{figure}

\begin{figure}[H]
\centering
\begin{subfigure}{.49\textwidth}
\includegraphics[width=\linewidth]{prva/omege.pdf}
\end{subfigure}
\caption*{Tukaj sem iskal optimalen $\omega$ za najhitrejšo konvergenco in primerjal z pričakovanim, ki je $\omega = 2/(1+\alpha \pi/J)$. $\alpha$ bi za kvadratno mrežo moral biti 1.}
\end{figure}




\begin{figure}[H]
\centering
\begin{subfigure}{.49\textwidth}
\includegraphics[width=\linewidth]{prva/profilkvadratlanski.pdf}
\end{subfigure}
\begin{subfigure}{.49\textwidth}
\includegraphics[width=\linewidth]{prva/profilkvadrat.pdf}
\end{subfigure}
\caption*{Na levi je profil iz lanskega praktikuma, na desni pa profil pridobljen s Chebyshevim pospeškom. Oblika in vrednosti sovpadata. Lani očitno pri risanju slike nisem postavil aspect razmerja osi na 1.}
\end{figure}

Ker lahko to hitro naredimo, preverimo še za pretok skozi okroglo cev, kjer je C po definiciji enak 1.

\begin{figure}[H]
\centering
\begin{subfigure}{.49\textwidth}
\includegraphics[width=\linewidth]{prva/profilkrog.pdf}
\end{subfigure}
\begin{subfigure}{.49\textwidth}
\includegraphics[width=\linewidth]{prva/krogomege.pdf}
\end{subfigure}
\caption*{Parameter alfa je torej v tem primeru nekoliko manjši. Pridobljen koeficient je $C=1.0046158152102151$, kar je že zelo blizu pravi vrednosti. Na desni vidimo, da se pri omega blizu 2 metoda hitro poslabša.}
\end{figure}


\begin{figure}[H]
\centering
\begin{subfigure}{.49\textwidth}
\includegraphics[width=\linewidth]{prva/polkrogomege.pdf}
\end{subfigure}
\caption*{Podoben graf kot prej še za polkrožen profil.}
\end{figure}


\begin{figure}[H]
\centering
\begin{subfigure}{.49\textwidth}
\includegraphics[width=\linewidth]{prva/profil.pdf}
\end{subfigure}
\begin{subfigure}{.49\textwidth}
\includegraphics[width=\linewidth]{prva/profilpolkrog.pdf}
\end{subfigure}
\caption*{Na levi lanski profil, lani izračunan C je bil 0.758. Na desni zdajšnji profil z C=0.779. Številske vrednosti se kar razlikujejo, vendar sem bil tukaj bolj hiter in neustrezno diskretiziral mrežo, kot je videno na sliki.}
\end{figure}

\begin{figure}[H]
\centering
\begin{subfigure}{.49\textwidth}
\includegraphics[width=\linewidth]{prva/profilmacka.pdf}
\end{subfigure}
\begin{subfigure}{.49\textwidth}
\includegraphics[width=\linewidth]{prva/mackaomege.pdf}
\end{subfigure}
\caption*{Še profil skozi cev v obliki mačke, izračunan C je 0.34585.}
\end{figure}
\newpage

\section{Toplotni profil}

Naša naslednja naloga je izračunati toplotni profil v enakostraničnem valju, kjer na spodnji ploskvi grejemo s konstantnim toplotnim tokom, zgornjo plosco pa izoliramo. Plašč držimo pri konstantni temperaturi. Imamo torej mešanico Dirichletovih in Neumannovih robnih pogojev. Z Dirichletovimi smo se spopadli že v prvi nalogi, za Neumannove pa bom enostavno odvod aproksimiral z diferenco in zahteval, da je enak določenem številu.

Ker gledamo stacionaren profil brez izvorov, rešujemo tokrat Laplacovo enačbo:
\begin{align*}
&\nabla^2 T = 0 \\
&\frac{\partial^2 T}{\partial r^2} + \frac{1}{r} \frac{\partial T}{\partial r} + \frac{\partial^2 T}{\partial z^2} = 0 
\end{align*}

Po diskretizraciji odvodov dobimo enačbo
\begin{align*}
&\left(1+\frac{1}{2j}\right) T_{j+1,k} + \left(1-\frac{1}{2j}\right) T_{j-1,k} + \frac{1}{4} T_{j,k+1} + \frac{1}{4} T_{j,k-1} - \frac{5}{2} T_{j,k} = 0 \\
&a_{j,k} T_{j+1,k} + b_{j,k} u_{j-1,k} + c_{j,k} u_{j,k+1} + d_{j,k} u_{j,k-1} + e_{j,k} u_{j,k} = f_{j,k} 
\end{align*}
Naša PDE je še vedno eliptična, torej lahko uporabimo naslednjo iteracijsko shemo:
\begin{align*}
&T_{j,k}^{n+1} = T_{j,k}^{n} + \omega (\tilde{T}_{j,k}^{n}-T_{j,k}^{n}) \\
&\tilde{T}_{j,k} = \frac{1}{e_{j,k}} ( f_{j,k} - a_{j,k} T_{j+1,k} - b_{j,k} T_{j-1,k} - c_{j,k} T_{j,k+1} - d_{j,k} T_{j,k-1}) \\
&\tilde{T}_{j,k} = \frac{2}{5} \left( \left(1+\frac{1}{2j}\right) T_{j+1,k} + \left(1-\frac{1}{2j}\right)T_{j-1,k} + \frac{1}{4} T_{j,k+1} + \frac{1}{4} T_{j,k-1} \right)
\end{align*}

Neumannov robni pogoj se potem enostavno glasi(npr, če je spodnji del valja pri k):
\begin{equation*}
(T_{j,k+1}-T_{j,k-1})/(2h) = K 
\end{equation*}
Pola pri j=0 se bom rešil na naslednji način: Za koordinate r definiram: $r_j = h*j + h/2$. To nas tudi reši levega robnega pogoja(pri r=0).
S takim izborom koordinat težje natančno ustrežemo robnemu pogoje, da je na plašču(ki s temi koordinatami ni dostopen) določena temperatura. Vendar pa bi bilo logično, da pri dovolj majhnem h to ne bo težava. Poleg tega rečem, da je moj valj v nekakšnem rezervoarju, kjer držim robno temperaturo fiksno na nič.

\begin{figure}[H]
\centering
\begin{subfigure}{.8\textwidth}
\includegraphics[width=\linewidth]{valj.pdf}
\end{subfigure}
\caption*{Na x osi je koordinata r, na y osi pa koordinata z. Pri tako majhnem toku je razlika med zgornjo in spodnjo ploskvijo komaj opazna.}
\end{figure}

\begin{figure}[H]
\centering
\begin{subfigure}{.8\textwidth}
\includegraphics[width=\linewidth]{valj2.pdf}
\end{subfigure}
\caption*{Pri večjem toku je razlika bolj vidna.}
\end{figure}

\begin{figure}[H]
\centering
\begin{subfigure}{.8\textwidth}
\includegraphics[width=\linewidth]{valj4.pdf}
\end{subfigure}
\caption*{Robni pogoj na plašču, kjer je zgornja polovica na dvakrat višji temperaturi kot spodnja povzroči veliko bolj zanimiv profil.}
\end{figure}

\begin{figure}[H]
\centering
\begin{subfigure}{.8\textwidth}
\includegraphics[width=\linewidth]{valj3.pdf}
\end{subfigure}
\caption*{Večji tok še dodatno zlomi simetrijo}
\end{figure}
\begin{figure}[H]
\centering
\begin{subfigure}{.8\textwidth}
\includegraphics[width=\linewidth]{omege.pdf}
\end{subfigure}
\caption*{Še odvisnost (Frobeniusove) norme razlike matrik zadnjih dveh iteracij od omege.}
\end{figure}
\end{document}